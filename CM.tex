\documentclass[12pt,a4paper]{report}
\usepackage[utf8]{inputenc}
\usepackage[french]{babel}
\usepackage[T1]{fontenc}
\usepackage{amsmath}
\usepackage{amsfonts}
\usepackage{amssymb}
\usepackage{graphicx}
\author{Malo Kerebel}

\usepackage{hyperref}
\hypersetup{
    colorlinks,
    citecolor=black,
    filecolor=black,
    linkcolor=black, urlcolor=black
}

\begin{document}


\begin{titlepage}

\centering{
	
	{\scshape\LARGE Université de Bretagne Occidentale \par}
	\vspace{1cm}
	{\scshape\Large Note de cours\par}
	\vspace{1.5cm}
	{\huge\bfseries Physique statistique\unskip\strut\par}
	\vspace{2cm}
	{\Large\itshape Malo Kerebel \par}
	\vfill
	Cours par\par
	Jean-Philippe \textsc{Jay}

	\vfill

% Bottom of the page
	{\large Semestre 6, année 2020-2021 \par}
}
\end{titlepage}

\tableofcontents

\chapter{Introduction générale}
\begin{center}
\textbf{CC1 (2021-01-12)}
\end{center}

physique statistique : étude du mouvement de gaz à l'échelle macroscopique.
Différences entre fermion et boson : leur spin, fermion spin demi entier (eg. les électrons), boson particule à spin entier (eg photon)\par
\quad \par
Bibliographie:
Physique Statistique, B. Diu, C. uthmann, D.Lederer\\
Physique Statistique, H. Ngp, C. Ngo\par
\quad \par

\section{Buts de la physique statistique}
Unifier le macroscopique et le microscopique, au $\text{XIV}^{\text{\`eme}}$ siècle, on a la thermodynamique, la mécanique et l'électro-magnétisme mais rien qui relie les uns aux autres.

Au niveau microscopique on a $\approx 10^{23}$ paramètres (de l'ordre du nombre d'Avogadro), auquel il faut avoir la vitesse et la position, il est impossible d'appliquer les résultats de la mécanique macroscopique dessus.
La physique statistique a donc pour but d'expliquer les comportements collectifs, de particules mais les résultats peuvent s'étendre à des réseaux de neurones ou des comportements de foules.

\section{Combinatoire}
Le dénombrement des objets ou des configuration. 2 système indépendant A et B, ayant $\Omega_a$ et $\Omega_b$ configurations, il y a \(\Omega_a \times \Omega_b\) configurations possible pour la juxtaposition de A et B.

\begin{center}
\textbf{CC2 (2021-01-14)}
\end{center}

Le nombre de permutations de N objets parmi N est N!

Le nombre de combinaison sans répétition s'obtient avec :
\[
	C^k_n = \dfrac{n!}{k!(n-k)!}
\]

\section{Probabilités - Statistiques}

événement aléatoire : résultat possible d'une expérience

Variable aléatoire : variable qui peut prendre l'une quelconque de ses valeurs possibles, inconnue d'avance. Discrète il y a un nombre finie de valeur, continue il y a un nombre infinie de valeur possible

\paragraph{Propriétés}
\begin{enumerate}
	\item \(0 \leq P_m \leq 1 \quad \forall m\)
	\item \(\sum_m P_m = 1\) (normalisation)
\end{enumerate}

Pour une variable aléatoire continue on utilise la densité de probabilité, qu'une la variable \(\in [x, x +\delta x]\)
\[
	w(x) = \lim_{\delta_x \rightarrow 0^+} \frac{\delta P(x)}{\delta_x}
\]
\[
	dP(x) = w(x)dx = \lim_{N \rightarrow \infty} \frac{dN(x)}{N}
\]
De même il y a la normalisation :
\[
	\int_{-\infty}^{+\infty} w(x)dx = 1
\]

Quand on est dans les bonne conditions :
\[
	P(e_1 ou e_2) = P(e_1) + P(e_2)
\]
\[
	P(e_1 et e_2) = P(e_1)\cdot P(e_2)
\]

L'écart quadratique moyen, ou variance, caratérise la dispersion de la distribution statistique, il est défini par :
\[
	(\Delta_f)^2 = \overline{(f - \overline{f})^2} = \overline{f^2} - (\overline{f})^2
\]
De même on définit l'écart-type \(\sigma\) comme la racine carré de la variance :
\[
	\sigma = \sqrt{(\Delta_f)^2} = \Delta_f
\]

La moyenne de résultat est :
\[
	\overline{n} = pN
\]



\end{document}